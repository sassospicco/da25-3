\section{Results}

Below the different results for the tests mentioned in previous chapters are shown. To save space only one result per test is shown.

\subsection{All loyal test}
Table \ref{table:resultsAllLoyal} shows the result of a test run with all loyal processes. Because of the small number, consensus is reached in the first round.


\begin{table}[h]
	\begin{tabular}{ | c | c  | c  | c| }
		\hline
  		round & process 1 & process 2 & process 3 \\
		\hline
 		intial value &  false & false & true  \\
		\hline
  		decision  & false & false & false  \\
		\hline
	\end{tabular}
	\caption{results for the test where all processes are loyal}	
	\label{table:resultsAllLoyal}
\end{table}

\subsection{5f < n test}
Table \ref{table:resultsFewDisloyal} shows the result of a test run with only 1 disloyal process.
The results show that the algorithm does reach consensus for this condition in 4 rounds.


\begin{table}[h]
	\begin{tabular}{ | c | c  | c  | c| c | c | c | c |}
		\hline
  		round & process 1 & process 2 & process 3 & process 4 & process 5 & process 6 & process 7 \\
		\hline
 		intial value &  true & false & true  & false & false & true & true \\
		\hline
		1  &  true & true & true  & true & false & false & false \\
		\hline
		2  &  false & true & false  & false & true & true & true \\
		\hline
		3  &  true & true & false  & true & true & true & true \\
		\hline
  		decision  &  true & true & true  & true & true & true & true  \\
		\hline
	\end{tabular}
	\caption{results for the test where 1 process is disloyal, compared to 7 loyal processes}	
	\label{table:resultsFewDisloyal}
\end{table}

\subsection{5f = n test}
Table \ref{table:resultsBoundaryDisloyal} shows the result of a test run where the condition 5f < n is just not met. 
Now the algorithm still reaches a consensus, though when we run more tests not always is the outcome correct.
Some test runs do not even reach consensus in a reasonable amount of time, or some show that 1 process reaches a decision without the others reaching it in the next round. 

The results will depend on the initial conditions a lot, if (almost) all processes start with the same value, chances are they still reach the right decision. When the possible intial values are more equally distributed, however, the faulty processes can make the intermediate decisions undecided or even tip over to the wrong decision.

\begin{table}[h]
	\begin{tabular}{ | c | c  | c  | c| c | c |}
		\hline
  		round & process 1 & process 2 & process 3 & process 4  \\
		\hline
 		intial value &  true & true & false & true\\
		\hline
		1  &  false & false & false  & true  \\
		\hline
		2  &  no random value &  no random value & no random value &  no random value \\
		\hline
  		decision  &  false & false & false  & false  \\
		\hline
	\end{tabular}
	\caption{results for the test where 1 process is disloyal, compared to 4 loyal processes}	
	\label{table:resultsBoundaryDisloyal}
\end{table}

\subsection{5f > n test}
 
When the number of disloyal processes is increased to be a significant amount of the total number of processes the algorithm most of the time doesn't reach consensus within a reasonable amount of time. 

This is probably due to the fact that when a large number of processors is faulty (f is high) the treshold for making a decision rises and the chance the process will create a new random intermediate decision rises as well. this way the chance of all processes reaching consensus gets really small.


\subsection{Large n test}
Table \ref{table:resultsLargeN} shows the result for tests where n gets increasingly large, to save on data we only show the amount of rounds it took to reach consensus for the different process counts.



\begin{table}[h]
	\begin{tabular}{ | c | c  | c | c |  c |}
		\hline
  		 number of processes & \multicolumn{3}{|c|}{Rounds untill consensus }  & average\\
		\hline
 		50 & 141 & 74 & 108  & 108   \\
		\hline
		60  & 147  &576 & 1077  & 600     \\
		\hline
		70  & 4136 & 2844 & 1529  & 2836    \\
		\hline
	\end{tabular}
	\caption{results for tests with increasingly larger N}	
	\label{table:resultsLargeN}
\end{table}

The results are too few to be an exact representation of the expected value, but they do show the exponential relation between number of processes and rounds to reach consensus.
