\section{Results}

\subsection{All loyal test}
table \ref{table:resultsAllLoyal} shows the result of a test run with all loyal processes. because of the small number, consensus is reached in the first round.


\begin{table}[h]
	\begin{tabular}{ | c | c  | c  | c| }
		\hline
  		round & process 1 & process 2 & process 3 \\
		\hline
 		intial value &  false & false & true  \\
		\hline
  		decision  & false & false & false  \\
		\hline
	\end{tabular}
	\caption{results for the test where all processes are loyal}	
	\label{table:resultsAllLoyal}
\end{table}

\subsection{5f < n test}
table \ref{table:resultsFewDisloyal} shows the result of a test run with only 1 disloyal process.
The results show that the algorithm does reach consensus for this condition in 4 rounds.


\begin{table}[h]
	\begin{tabular}{ | c | c  | c  | c| c | c | c | c |}
		\hline
  		round & process 1 & process 2 & process 3 & process 4 & process 5 & process 6 & process 7 \\
		\hline
 		intial value &  true & false & true  & false & false & true & true \\
		\hline
		1  &  true & true & true  & true & false & false & false \\
		\hline
		2  &  false & true & false  & false & true & true & true \\
		\hline
		3  &  true & true & false  & true & true & true & true \\
		\hline
  		decision  &  true & true & true  & true & true & true & true  \\
		\hline
	\end{tabular}
	\caption{results for the test where 1 process is disloyal, compared to 7 loyal processes}	
	\label{table:resultsFewDisloyal}
\end{table}

\subsection{5f = n test}
table \ref{table:resultsBoundaryDisloyal} shows the result of a test run where the condition 5f < n is just not met (5f = n). 
Now the algorihm reaches a consensus, but not the expected one given the intial values.


\begin{table}[h]
	\begin{tabular}{ | c | c  | c  | c| c | c |}
		\hline
  		round & process 1 & process 2 & process 3 & process 4 & process 5  \\
		\hline
 		intial value &  true & true & false  & false & true\\
		\hline
		1  &  false & false & false  & false & true  \\
		\hline
		2  &  no new value & no new value & no new value  & no new value & no new value \\
		\hline
  		decision  &  false & false & false  & false & false \\
		\hline
	\end{tabular}
	\caption{results for the test where 1 process is disloyal, compared to 5 loyal processes}	
	\label{table:resultsBoundaryDisloyal}
\end{table}

\subsection{5f > n test}
 
When the number of disloyal processes is  increased to be a signifacnt amount of the total number of processes the algorithm most of the time doesn't reach consensus within a reasonable amount of time.

\subsection{Large n test}
table \ref{table:resultsBoundaryDisloyal} shows the result of a test run where the condition 5f < n is just not met (5f = n). 
Now the algorihm reaches a consensus, but not the expected one given the intial values.


\begin{table}[h]
	\begin{tabular}{ | c | c  | c  | c| c | c |}
		\hline
  		round & process 1 & process 2 & process 3 & process 4 & process 5  \\
		\hline
 		intial value &  true & true & false  & false & true\\
		\hline
		1  &  false & false & false  & false & true  \\
		\hline
		2  &  no new value & no new value & no new value  & no new value & no new value \\
		\hline
  		decision  &  false & false & false  & false & false \\
		\hline
	\end{tabular}
	\caption{results for the test where 1 process is disloyal, compared to 5 loyal processes}	
	\label{table:resultsBoundaryDisloyal}
\end{table}